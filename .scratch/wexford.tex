\documentclass[11pt]{article}
\usepackage[margin=1in]{geometry}
\geometry{a4paper}
\usepackage{graphicx}
\usepackage{float}
\usepackage{wrapfig}
\usepackage{url}
\usepackage{tipa}
\usepackage{hyperref}
\usepackage{apacite}

\usepackage{natbib}

\linespread{1.5}

\graphicspath{{Pictures/}}

\begin{document}

\begin{titlepage}

\newcommand{\HRule}{\rule{\linewidth}{0.5mm}}

\center

\begin{minipage}{0.8\textwidth}
\begin{flushleft} \large
Jim \textsc{O'Regan}
\\
{\large 15314318} \\[1.5cm]

{ \large \bfseries An analysis of the vowel system of the Wexford variety of Hiberno-English, and a comparison with Southern Michigan.}\\[1.5cm]

M.Phil. in Speech and Language Processing
\\
%{\large \today}\\[3cm]
{\large Speech Production, Hearing, and Perception} \\
{\large Michaelmas Term, 2016}
\end{flushleft}
\end{minipage}

\begin{minipage}[t][10cm][b]{0.5\textwidth}
\begin{flushleft} \large
Wordcount: 
\\
\end{flushleft}
\end{minipage}


\vfill

\end{titlepage}

\section{Introduction}

...will be compared with Southern Michigan~\citep{hillenbrand2003american} blah~\citep{hillenbrand1995acoustic}

vowels in a /hVd/ context, based on \cite[p. 175]{peterson1952control}

\section{Methods}

\section{Materials}

\section{Results}

\section{Introduction}

\subsection{Vowels of Wexford English in the non-rhotic environment}

\subsubsection{Monophthongs}

\subsubsection{Diphthongs}

\subsubsection{Role of quality and duration in the vowel system: selected vowel pairs}

\subsection{Vowels of Wexford English in the rhotic environment}

\subsection{A comparison of Wexford English with American English}

\section{Discussion}

\section{Conclusions}

\section{References}

\bibliographystyle{plainnat}
\bibliography{sphp.bib}

\section{Appendices}

\end{document}
