
Despite the availability of rule-based tools for the morphological analysis and tagging of Irish~\citep{elaine09}, I chose to train a set of statistical models for the tagging and lemmatisation of Irish. The primary motivation behind this was to follow the workflow most amenable to practitioners of statistical NLP, who tend to prefer not to use rule-based tools where possible.



\subsection{Multiwords}

The Irish Dependency Treebank, having been tagged with 

Splitting the multiwords into their individual words has posed some difficulties: in a number of cases, the multiwords contain calcified words which no longer have meaning in modern Irish. For example, ``moite'' in the phrase ``c\'e is moite'' (\textit{except}) is absent from dictionaries of the modern language, except as part of that phrase\footnote{eDIL (\href{http://www.dil.ie/32548}{http://www.dil.ie/32548}) lists it as a comparative of ``m\'or''}. 
!!!CHECK
``dt\'i'' in the phrase ``go dt\'i'' is a calcified present subjunctive form of the verb ``tar'', but can legitimately be analysed in the modern language as the eclipsed forms of two nouns which share the lemma ``t\'i'': a feminine noun meaning \textit{line} or \textit{track}, and a masculine noun meaning \textit{tee}.

Political parties: Prop:Noun:Masc --> Sinn Fein

